\section*{Abstract}

Advances in block-chain technology has not only elevated the price of GPUs, but also energized the creative community. First seen in 2014 \cite{quantum}, the Non-fungible Token (NFT) bring to life the uniqueness of masterpiece created by artists. With the aids of block-chain technology, it is finally possible to create digital artworks that is distinguishable from the others. 


The ideas adopted in NFT has greatly pushed the progression of art digitization, while it does not necessarily adopt block-chain based solution. Our system, DAP\footnote{Our code are publicly available in: \url{https://github.com/Yb-Z/COMP3334-Group-Project}}, is a platform representing representing, storing and exchanging digital artwork. Besides, it is also designed to protect and exchange ownership of digital artwork. Note that, the platform aims to protect the ownership of the artwork, instead of the artwork itself. The art work is very difficult to protect since it is in digital format, and you can simply take an screenshot and it's gone. In this report, we aim to analyze general-sense digital art platform in terms of security. We will discuss the security requirements of the platform, and we will provide case studies to similar platforms on the market. In second half of this report, we will specify the design considerations of the platform, and we will also provide a detailed deploy guide in the last section.

